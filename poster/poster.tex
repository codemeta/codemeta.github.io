\documentclass[final,hyperref={pdfpagelabels=false},xcolor=svgnames]{beamer}

\mode<presentation>
  {
  \usetheme{Boadilla}
  %\usetheme{Frankfurt}
  }
  \newcommand{\E}{\mathrm{E}} 
  \newcommand{\ud}{\mathrm{d}} 
  \newcommand{\degree}{$^{\circ}$}
  \newcommand{\tb}[1]{\textcolor{blue}{#1}} 

  \bibliographystyle{evolution}
  \usepackage{times}
  \usepackage{amsmath,amsthm, amssymb, latexsym}
  \usepackage{graphicx}
  %\boldmath


  \usepackage[english]{babel}
  \usepackage[latin1]{inputenc}
  \usepackage[orientation=landscape,size=E,scale=1.4,debug]{beamerposter}

%	\usepackage{rotating}


\graphicspath{{/home/cboettig/Documents/ucdavis/research/presentations/images/}}

  %%%%%%%%%%%%%%%%%%%%%%%%%%%%%%%%%%%%%%%%%%%%%%%%%%%%%%%%%%%%%%%%%%%%%%%%%%%%%%%%%5
  \title[]{\VeryHuge \textbf{Detecting Evolutionary Regime Shifts $\phantom{longword}$ }}
  \author{Carl Boettiger}
  \institute{Center for Population Biology, University of California, Davis}
  \usecolortheme[named=DarkBlue]{structure} 

  \graphicspath{{/home/cboettig/Documents/ucdavis/research/presentations/images/}}

\setbeamertemplate{background canvas}{\includegraphics
	[width=\paperwidth,height=\paperheight]{fish/csgf_2011.jpg}}

%%%%%%%%%%%%%%%%%%%%%%%%%%%%%%%%%%%%%%%%%%%%%%%%%%%%%%%%%%%%%%%%%%%%%%%%%%%%%%%%%5
  \begin{document}
  \begin{frame}[t] 
%  \maketitle

\frametitle{\VeryHuge{\textbf{ Detecting Evolutionary Regime Shifts $\phantom{longword}$ } }\\ \LARGE{\hspace{1cm} Carl Boettiger, UC Davis}}
\vspace{4cm}


%%%%%%%%%%%%%%%% Two top columns %%%%%%%%%%%%%%%

\begin{columns}[t] % LEFT COLUMN 
\begin{column}{.20\paperwidth}
{\LARGE Abstract }
The genomic revolution has made a wealth of genetic data available over the past
decade that has opened up a third avenue and provided a periscope into past origins of
today's diversity. Phylogenetic trees, reconstructed from patterns of DNA in extant
species, provide a map of how these species have evolved from their common ancestry.
In my work on these evolutionary origins, I am developing methods that combine
phylogenetic and present-day trait information across a range of taxa to look back into
their past and identify key evolutionary transitions responsible for the diversity we see
today.
  \vspace{1cm}

\begin{center}
{\LARGE A Simple Model} 

  $$  \ud X = \underbrace{\alpha}_{\textrm{selection}} (\underbrace{\theta}_{\textrm{optimum}} - X) \ud t + \underbrace{\sigma}_{\textrm{divers. rate}} \ud B_t  $$

  \includegraphics[width=.15\paperwidth]{evo2011/combined_sketch3.pdf}

%\vspace{1cm}

% \includegraphics[width=.2\paperwidth, height=6in]{evo2011/brownie}\hspace{1cm} \includegraphics[width=.1\paperwidth, height=6in]{evo2011/ouch}

 %  Differing rates of diversification | Differing selective optima
  \end{center}



\end{column}

\begin{column}{.2\paperwidth} 
	{\LARGE Results }

 % \begin{block}{}
  \begin{center}
    Trait Diversification Rate Model 
    \href{http://www.flickr.com/photos/cboettig/5842718829}{
    \includegraphics[width=\textwidth]{evo2011/protrusion_sigmas.png}}

    Stabilizing Selection Strength 
    \href{http://www.flickr.com/photos/cboettig/5842718565}{
    \includegraphics[width=\textwidth]{evo2011/protrusion_alphas.png}}

  \end{center}
%  \end{block}

  \begin{block}{So which is the better model?}
  \begin{columns}
    \column{.5\linewidth}
  \begin{itemize}
    \item Differing Diversification Log Likelihood: \textbf{-94} \\  Parameters: 4
    \item Differing Selection Log Likelihood: \textbf{-92} \\Parameters: 4
  \end{itemize}
  \column{.5\linewidth}
    \begin{center}
      \href{http://www.flickr.com/photos/cboettig/5842718351}{
      \includegraphics[width=\textwidth]{evo2011/protrusion_alphas_v_sigmas.png}}
    \end{center}
  \end{columns}
  \vspace{1cm}
  Not enough power for this comparison in this trait.\\  
  \end{block}

\vspace{1.9in}
{\Large Computations on treacherous surfaces}
\large
\begin{itemize}
  \item Likelihood Ridges: Frequent trade-offs in parameter combinations
  \item Rugged surfaces: assessing optimization algorithm convergence)
  \item Unreliable asymptotic results (standard information criteria)
\end{itemize}
\end{column}

\begin{column}{.2\paperwidth} % RIGHT COLUMN

%\begin{block}{}
  \begin{center}
    Differing Stabilizing Selection Strength
    \href{http://www.flickr.com/photos/cboettig/5841932378}{
    \includegraphics[width=\textwidth]{evo2011/labrid_alpha.png}}

    Differing Trait Diversification Rate
    \href{http://www.flickr.com/photos/cboettig/5841932492}{
    \includegraphics[width=\textwidth]{evo2011/labrid_sigma.png}}
  \end{center}
%\end{block}

%\begin{block}{}
  \begin{center}
    \href{http://www.flickr.com/photos/cboettig/5841932260}{
    \includegraphics[width=.75\textwidth]{evo2011/labrid_compare.png}}
  \end{center}
%\end{block}

\vspace{2in}

{\Large Visualizing regime shifts in simple models:}

 \includegraphics[width=4in, height=4in]{evo2011/brownie}\hspace{1cm} 
\includegraphics[width=4in, height=4in]{evo2011/ouch}

\end{column}



\begin{column}{.28\paperwidth} % PLACEHOLDER FOR BACKGROUND
  \vspace{9in}
  \begin{center}
    {\LARGE Bayesian version by MCMC}
    \href{http://www.flickr.com/photos/cboettig/5844577056}{
  \includegraphics[width=\textwidth]{evo2011/mcmc_labrid_alpha.png}}
  \end{center}



  {\LARGE Summary}
  \begin{enumerate}
    \item Introduce phylogenetic method to detect evolutionary regime shifts.
    \item Highlight pitfalls of inadequate infromation.
    \item Introduce a method to quantify power and uncertainty of estimates.
    \item Working with limits: offering smaller sub-models for smaller data sets.
  \end{enumerate}
		
  \vspace{.5in}

   {\LARGE Acknowledgements} \\
		\vspace{1cm}
		\normalsize{
    \begin{itemize}
      \item P. Wainwright
      \item G. Coop
      \item P. Ralph
      \item S. Price
    \end{itemize}

		  \includegraphics[width=4in]{icons/csgf.png}

      Development version 
\href{https://github.com/cboettig/wrightscape}{https://github.com/cboettig/wrightscape}

\vspace{-5in} \begin{flushright}\includegraphics[width=3in]{evo2011/giturl}\end{flushright}
		}

\end{column}
\end{columns}

\end{frame}
\end{document}
%% Fontsize commands
	%      \tiny
	%      \scriptsize
	%     \footnotesize
	%     \normalsize
	%      \large 
	%      \Large
	 %     \LARGE
	%      \veryHuge
	%      \VeryHuge
	 %     \VERYHuge

